%!TEX TS-program = xelatex
\documentclass[]{friggeri-cv}
\addbibresource{bibliography.bib}

\begin{document}
\header{frank}{singel} 
       {Software Engineer}


% In the aside, each new line forces a line break
\begin{aside}
  \section{about}
    Frank Singel
    La Jolla, CA
    92037
    ~
    724-431-8661
    ~
    \href{mailto:fjs52@case.edu}{fjs52@case.edu}
  \section{programming}
    {\color{red} $\varheartsuit$} Java
    JavaScript
    (d\={o}j\={o})
    HTML, SCSS
    Python, SQL
    Groovy
  \section{libraries}
    Maven, 
    Hibernate ORM, Ant
    jUnit (BDD Mockito)
    Log4j, ActiveMQ
  \section{other skills}
    Git, Eclipse, IntelliJ
    AppScan, Fortify
    Burp Suite
    Thread/Heap Dump Analysis,
    Agile Methodologies, PSIRT
    GCC/PTP
  \invisiblesection{cloud aws node.js}
    
\end{aside}

\section{summary}
I'm a primarily-backend engineer with experience in writing performant web applications with a penchant for secure coding practices. I'm also currently exploring the market.

\section{education}

\begin{entrylist}
  \entry
    {2010–2014}
    {B.S. in Computer Science}
    {Cleveland, OH}
    {Case Western Reserve University}
\end{entrylist}

\section{experience}

\begin{entrylist}
  \entry
    {2019 current}
    {ServiceNow, San Diego}
    {Software Engineer.}
    {\emph{
    •	Wrote performance instrumentation for our Data Streaming product.\\*
    •	Worked with Product Management and across teams to Story out an epic for a major release.\\*
    •	Created and presented documentation slides and demos for new features to whole support org each release.\\*
    •	Regularly presented demos of my feature work to entire support organization. \\*
    %Do I talk about Data Stream Quick Mode? Need perf metrics.
    %How do I talk about team One Note?
    •	Spurred team collaboration and adoption of team-specific documentation. \\*
    }}
  \entry
    {02-07-2015}
    {UrbanCode, Cleveland}
    {Software Engineer.}
    {\emph{
    •	Delivered new features and helped develop tests for our on-prem \\* DevOps management web app on a RESTful Java/JS tech stack.\\*
    % Oops, forgot what this was. Uncomment if you can remember
    %•   Designed, planned, and implemented a refactor of a popular feature to \\* improve network performance from O(n\textsuperscript{2}) to O(n).\\*
    •	Worked closely with customers to resolve time sensitive and impactful\\* issues. This required rapid familiarization with many tech stacks regularly.\\*
    •	Trained and upskilled a fresh offshore development team into productivity within 3 months of their hire and incorporated them into our scrums. \\*
    •	Performed a comprehensive audit on our product to locate and purge Cross Site Scripting vulnerabilities from it through various automated means.\\*
    •	Owned responsible disclosure and resolution of our security vulnerabilities.}}
  \entry
    {05-08-2014}
    {MIM Software, Cleveland}
    {Software Development Internship.}
    {\emph{
    •	Developed and improved new features for MIM's voxel-based medical imaging software. Also was responsible for defect resolution in an Agile development environment working primarily with Java, Swing, and JUnit.}}
  \entry
    {05–08-2013}
    {Rockwell Automation, Cleveland}
    {Software Engineering and Test Internships.}
    {\emph{
    •	Responsible for development, testing, and defect resolution of new and existing features for Rockwell’s flagship Studio5000 industrial control system design and configuration software.  Experiences included use of C++, perl, HTML, Javascript, and OS Virtualization.}}
\end{entrylist}

\section{interests}
•	Case Western Reserve Ice Hockey Team
•	Magic: The Gathering L2 Judge
\\*•	Theta Chi Fraternity
•	Playing with pentesting tools
•	James Mickens
%%% This piece of code has been commented by Karol Kozioł due to biblatex errors. 
% 
%\printbibsection{article}{article in peer-reviewed journal}
%\begin{refsection}
%  \nocite{*}
%  \printbibliography[sorting=chronological, type=inproceedings, title={international peer-reviewed conferences/proceedings}, notkeyword={france}, heading=subbibliography]
%\end{refsection}
%\begin{refsection}
%  \nocite{*}
%  \printbibliography[sorting=chronological, type=inproceedings, title={local peer-reviewed conferences/proceedings}, keyword={france}, heading=subbibliography]
%\end{refsection}
%\printbibsection{misc}{other publications}
%\printbibsection{report}{research reports}
\end{document}
